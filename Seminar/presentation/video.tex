\documentclass[aspectratio=169]{beamer}
\usetheme{CambridgeUS}

\usepackage{color}
\usepackage{amsfonts}
\usepackage{comment}
\usepackage{moreverb}

%% Video stuff
\usepackage{media9}

\newcommand{\myurlshort}[2]{\href{#1}{\textcolor{gray}{\textsf{#2}}}}



%%% Start of the presentation
\title{Adding video in a beamer presentation}
\author{By \myurlshort{http://biostat.jhsph.edu/~lcollado/}{L Collado-Torres}}
\date{
December 5th, 2012
}

\begin{document}

\begin{frame}[allowframebreaks]
  \titlepage
\end{frame}

\begin{frame}[allowframebreaks]
  \tableofcontents
\end{frame}

%%%%%%%%%%%%%%%%%%%%%%%%%%%%%%%%%%%%%%%%%%%%%%%%%%%%%%%%%%%%%%%
\section{Sources}

\begin{frame}
  \frametitle{You might want to check}
  \begin{itemize}
  \item \myurlshort{http://pages.uoregon.edu/noeckel/PDFmovie.html}{PDF with animations} is a great post that ultimately points you to media9.
  \item \myurlshort{http://www.ctan.org/tex-archive/macros/latex/contrib/media9/}{media9 CTAN page}
  \item \myurlshort{http://ctan.mirrorcatalogs.com/macros/latex/contrib/media9/doc/media9.pdf}{media9 manual}: open it with Acrobat to see all the nice things.
  \item \myurlshort{http://makarandtapaswi.wordpress.com/2009/07/10/movie-package-movie15/}{Post about using movie15}: note that movie15 is deprecated.
  \end{itemize}
\end{frame}

%%%%%%%%%%%%%%%%%%%%%%%%%%%%%%%%%%%%%%%%%%%%%%%%%%%%%%%%%%%%%%%
\section{Test}

\begin{frame}
  \frametitle{Youtube video}
  \includemedia[
width=7.11cm,height=4cm,
activate=pageopen,
flashvars={
modestbranding=1 % no YT logo in control bar
&autohide=1 % controlbar autohide
&showinfo=0 % no title and other info before start
&rel=0 % no related videos after end
},
url % Flash loaded from URL
]{}{http://www.youtube.com/v/9bZkp7q19f0?rel=0}
\end{frame}

\begin{frame}
	\frametitle{What you need for media9}
Basically, the newest version of textlive and a new version of Acrobat Reader (above 9 something).
\end{frame}

\end{document}